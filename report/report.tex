% \k{a} \k{a}
% \'{c} \'{c}
% \k{e} \k{e}
% \l{} \l{}
% \'{n} \'{n}
% \'{o} \'{o}
% \'{s} \'{s}
% \.{z} \.{z}
% \'{z}\'{z}

% \k{a} \k{A}
% \'{c} \'{C}
% \k{e} \k{E}
% \l{} \L{}
% \'{n} \'{N}
% \'{o} \'{O}
% \'{s} \'{S}
% \.{z} \.{Z}
% \'{z}\'{Z}

\documentclass[10pt,a4paper]{article}
\usepackage[utf8]{inputenc}
\usepackage[margin=2cm]{geometry}
\usepackage{polski}
\usepackage{listings}
\usepackage[T1]{fontenc}
\usepackage{graphicx}

\begin{document}

\textbf{\LARGE Stanisław Kozioł}
\hspace{\stretch{1}}
Data wykonania: 2015-02-03
\vspace{0.2cm}
\hrule

\section{Wstęp}

\paragraph{}
Pappararagraf $f(x)=sin(x)$.

\section{Implementacja i wyniki}

\paragraph{}
\begin{lstlisting}
<ctime>
\end{lstlisting}

\paragraph{}
\begin{centering}
\begin{tabular}{|c|r|r|r|r|}
\hline
ilość wątków & średni czas obliczeń & odchylenie standardowe & przyśpieszenie \tabularnewline
\hline
 1 & 0.5 & 0.001 & 1.00\tabularnewline
\hline
\end{tabular}
\par\end{centering}

\begin{figure}[!hbp]
\begin{center}
  \includegraphics[width=0.75\textwidth]{time.eps}
  \caption{Komentarz wykresu czasów}
\end{center}
\end{figure}

\clearpage

\begin{figure}[!hbp]
\begin{center}
  \includegraphics[width=0.75\textwidth]{speadup.eps}
  \caption{Komentarz wykresu przyśpieszenia}
\end{center}
\end{figure}

\section{Wnioski}

\paragraph{}
Pappararagraf


\end{document}

